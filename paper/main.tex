\documentclass{proc}
\usepackage{graphicx}
\usepackage{mathtools}
\usepackage{fixltx2e}
\usepackage{algorithm}
\usepackage{algpseudocode}
\usepackage{caption}
\usepackage{url}
\usepackage{amsmath}
\usepackage{mathrsfs}

\title{
The Small World Network Effect in Open Source Project Teams
\author{Kevin Peterson\\
\small \texttt{pete1968@umn.edu}
}
}

\begin{document}
\maketitle

\begin{abstract}
Team cohesion and the dynamics of team forming are important parts of any project, with software projects being no exception. An interesting aspect of team building is the relationships formed between the team members. If follows that visualizing software team members as a graph can give some insight as to what the optimal team conditions are. As team members move between projects, these graphs gets more and more connected as team members make connections. We show that this connectivity, known as the "small world effect," has a positive impact on team performance at when the connectivity levels are moderate. This aligns with similar research findings of non-software teams. We do, however, find high performance at the extremes of the connectivity range.
\end{abstract}

\noindent \\\textbf{Keywords.} Git, GitHub, Open Source

\section{Introduction}


\subsection{Hypotheses}


\section{Methods}
\subsection{Collection and Storage}

\subsection{Analysis}

\section{Results}

\subsection{Quantitative}

\subsection{Qualitative}

\section{Conclusion}
\section{Acknowledgments}

\bibliographystyle{plain}
\bibliography{bibliography}


\end{document}
\clearpage

\section*{Appendix A: Interview Transcript}
Interview with Traci St. Martin, PMP\textregistered, Mayo Clinic, Nov 21, 2013.

\subsection*{Q1: How do development teams collaborate and share ideas?}
Co-location is important for teams, as it really fosters communication and collaboration. Being able to quickly talk with someone and resolve a problem is very important. In some cases, having the team in the same room is beneficial, especially if the project is going badly. That way, ideas can get discussed and problems resolved very quickly. Regular meetings are also important with large teams. With these meetings, it is important to have the correct tools to facilitate communication. These meetings help projects have visibility into each other. Management support is essential to this communication process. Meetings between teams can consist of all members of the teams if the teams are small enough, but most of the time should just be a representative.

\subsection*{Q2: Do teams generally remain intact over multiple projects, or do members often shuffle and re-combine?}
I have had both. It really depends on the projects and the situation at the time. It can be nice to have new people brought in to make things different, and to bring some new thoughts and ideas to the team. Even one person can change the dynamics of a team, either positively or negatively. As a project manager, I have to be watchful of this, as negativity can spread quickly. I have also noticed that as people move around to different projects, they are generally brought up to the level of the team. For example, if somebody joins a high performing team, they tend to elevate their skill to match that of the group, which is a good thing.

\subsection*{Q3: When assembling a team, which is preferred -- members familiar with each other, or members familiar with the task or project?}
Familiarity is good, but you have to account for burnout if people work on the same project for too long. This really depends on project dynamics, special skills needed, etc. It is important to note, however, that team members can be familiar with each other but not get along well, or not work well with each other. I have had projects where team members were very familiar with each other, but did not function well as a team. Clashing personalities and other team problems can cause a toxic environment. Most importantly, whether the team is familiar with each other or not, is that they can share knowledge and skills.

\subsection*{Q4: What are some characteristics of a high performing team?}
Most teams I notice go through the Forming, Storming, Norming and Performing stages. Good teams usually have a strong leader, a good vision, and good management support. Of course, they also possess the necessary skills for the project (or, if not, are willing to learn). Good teams have respect for each other, their leadership, and their management. Professionalism is also an indicator of a high performing team. Teams that truly care about their job, the team, and the project are generally more successful. Teams are generally more successful if the individuals themselves are skilled and talented. Being able to react to change is important, and good teams generally have good change management procedures.

\subsection*{Q5: In your experience, is it better to (A) keep an innovative team intact, (B) periodically bring in new members or re-assign existing members, or (C) split up and disperse the team to as many other projects as possible?}
In general, I like to keep good teams intact. It is, however, nice to bring in or add a new member to a team. One strategy I like to employ is to bring in a new member, train them and let them learn from the high performing team, and move them out of the project and into a new one. This keeps the core team intact, which is important. Priorities can greatly influence this, and sometimes personnel moves are driven from higher management priorities. If experts are removed from a high performing team, the team may suffer. There is, however, a risk that knowledge becomes concentrated. High priority/high visibility teams have a tendency to get the highest performing individuals, which can cause other projects to suffer. 
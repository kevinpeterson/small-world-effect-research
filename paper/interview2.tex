\clearpage

\section*{Appendix B: Interview Transcript}
Interview with M Iftekhar (Ifte) Rana, PMP\textregistered, Mayo Clinic, Nov 26, 2013.

\subsection*{Q1: How do development teams collaborate and share ideas?}
In my previous experiences, there were tight dependencies between teams and changes could have large impacts. Because of this, team collaboration was important and was stressed. Because of the tight dependencies, teams were involved in each other's planning phases. We found that bringing teams together during planning helped collaboration and communication. Also, we noticed that the quality of communication and collaboration between teams had a direct impact how how teams performed, and how the project faired as a whole.


\subsection*{Q2: Do teams generally remain intact over multiple projects, or do members often shuffle and re-combine?}
It is more of a problem when focus changes rapidly. If there is a stable focus for the project, and the goals of the project remain relatively intact, shifting of personnel has less impact. Continuously changing focus definitely has a negative impact on team performance. Also, when focus is changing rapidly, it is hard to maintain a stable team core, as they tend to get reassigned or loaned out to different projects.


\subsection*{Q3: When assembling a team, which is preferred -- members familiar with each other, or members familiar with the task or project?}
This really depends on the time-line of the project. Short-term projects don't have as much time for team bonding -- it is more important to get the people with the right skill set. For longer term projects, you have to be much more careful about people getting along with each other. For these types of projects, you can afford to invest some time into getting the team right and doing some team building. Also, with long term projects, you have more of a chance of members ``learning on the job," so even if the technical skills aren't there right away, you can factor in some learning time if they are a good fit personality-wise. It is the same reason why we ask a lot of behavioral questions for new full-time employees, and less so with contractors. It all depends on the time commitment and how much team building you can plan for.


\subsection*{Q4: What are some characteristics of a high performing team?}
The number one most important factor for a high performing team is good communication skills. Good teams that I've worked are highly connected, and all members of the team communication with each other. This leads to feedback coming from many different angles, which helps teams get better. Team performance is usually tied directly to how well a team communicates and shares knowledge with each other.


\subsection*{Q5: In your experience, is it better to (A) keep an innovative team intact, (B) periodically bring in new members or re-assign existing members, or (C) split up and disperse the team to as many other projects as possible?}
It is usually good to keep teams intact if they are performing well. That said, if you do not keep growing the team, they will quickly lose touch with changes in technology. New members help to bring this into the team, and if a team is isolated they often don't keep up. Technology trends move very quickly, so it is essential to have fast communication channels so that information can spread quickly. Even if new/different members aren't coming into the team, there still should be communication channels open. Communication between teams should happen whether or not teams are actually working together. In general, it is best to periodically bring in new members, but keep the core team intact if possible.